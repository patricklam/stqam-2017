\documentclass[10pt,hidelinks]{article}
\usepackage[letterpaper, hmargin=0.75in, vmargin=0.75in]{geometry}
\usepackage{graphicx}
\usepackage[hyphens]{url}
\usepackage{hyperref}
\usepackage{listings}
\usepackage{pgf}
\usepackage{courier}

\parindent 0in
\parskip 1.5ex


\lstset{ %
language=Java,
basicstyle=\ttfamily\scriptsize,commentstyle=\scriptsize\itshape,showstringspaces=false,breaklines=true}


\begin{document}

\title{
ECE453/CS447/SE465 \\
Software Testing, Quality Assurance, and Maintenance\\
Assignment/Lab 1, version 0}
\author{Patrick Lam \\
{Release Date:  January 16, 2017} \\
}
\renewcommand{\today}{}
\maketitle

\begin{center}

{\bf Due:  11:59 PM, Monday, January 30, 2017} \\
{\bf Submit: via ecgit }\\
\end{center}

% due after L12

%Please download a1-skeleton.tar.gz from the course repository to get the necessary source code and 
%test cases needed for finishing this assignment/lab. 

\section*{Getting set up}
After setting up your ssh key at \url{http://ecgit.uwaterloo.ca}, \\ fork the provided git repository at {\tt git@ecgit.uwaterloo.ca:se465/1171/a1}:

\begin{center}
{\tt ssh git@ecgit.uwaterloo.ca fork se465/1171/a1 se465/1171/USERNAME/a1}
\end{center}

\noindent and then clone the provided files:

\begin{center}
{\tt git clone git@ecgit.uwaterloo.ca:se465/1171/USERNAME/a1}
\end{center}

(You can also download the provided
files from the course repository ({\tt assignments/a1} in \url{https://github.com/patricklam/stqam-2017}),
but don't do that if you're in the course---it'll make submitting harder.)

An account on {\tt ecelinux.uwaterloo.ca} is available to you.
Several of the resources required for this assignment are already installed on these servers. Probably the Vagrant image is easiest to work with.
If you are attempting to connect to a server from off campus, remember you will need to connect to the University's VPN first: \url{https://uwaterloo.ca/information-systems-technology/services/virtual-private-network-vpn/about-virtual-private-network-vpn}

I expect each of you to do the assignment independently. I will follow UW's Policy 71 for all cases of plagiarism.
 
\section*{How to run everything}
Please see {\tt a1-technotes.pdf} for more details about how the provided software works.
\newpage
 \section*{Submission instructions:} 
Commit {\bf and push} your modifications back to your fork on {\tt
  ecgit}.  It's git, so you can submit multiple times. After
submission, {\bf please re-clone your submissions to make sure you
  have uploaded all necessary files}.
 
\section*{Submission summary}
Here's what you need to submit in your fork of the repo. Be sure to commit
and {\bf push} your changes back to {\tt ecgit}.
\begin{enumerate}
\item your modified {\tt FormattedCommandAliasTest.java} file in path
\url{shared/bukkit/src/test/java/org/bukkit/command}.
\item in directory {\tt q2}, either file {\tt exploratory.pdf} or {\tt exploratory.txt}, respectively
in PDF or text format. I've included {\tt exploratory.tex} which you can
\LaTeX into {\tt exploratory.pdf}. (No Microsoft Word files, please).
\item in directory {\tt q3}, file {\tt fix.diff} (generate with {\tt git diff}) 
and file {\tt failure.pdf}.
\item in directory {\tt q4}, file {\tt no-dead-end.diff} and
either file {\tt graph.png} or {\tt graph.pdf}.
\item in directory {\tt q5}, your scripts/test harness with fake data,
along with the code you use to generate it. (I used Python), along with 
file {\tt scalability.pdf} summarizing your timings.
\end{enumerate}
 
 \begin{center}
 \begin{tabular}{c|cc}
 Question   &  TA in Charge \\ \hline
1 &  \\ 
2 &  \\ 
3 &  \\ 
4 &  \\ 
 \end{tabular}
 \end{center}

\newpage

\section*{Question 1 (10 points)}
For this question, you will write JUnit tests for the {\tt
  FormattedCommandAlias} class of the Bukkit Minecraft server API to
achieve 100\% statement coverage.

The provided Vagrant image includes a slightly modified version of
Bukkit in the \url{~/shared/bukkit} directory (bukkit diff also available in course
github at \url{assignments/a1/bukkit-test-instrumentation.diff}.  
I added a skeleton test class for {\tt
  FormattedCommandAlias}, called {\tt FormattedCommandAliasTest}.

You can run the tests in the VM with the command {\tt mvn test}.  To
generate the Jacoco coverage report, use {\tt mvn package}. You'll
find the resulting reports in
\url{~/shared/bukkit/target/site/jacoco/org.bukkit.command}.

\paragraph{Your task.} Add JUnit unit tests to {\tt FormattedCommandAliasTest}
that achieve 100\% statement 
coverage for the {\tt org.bukkit.command.FormattedCommandAlias} class
and that verify the results of the computation.

Marking scheme: We will mark your modified {\tt
  org.bukkit.command.TestFormattedCommandAlias} class.  5 points for
coverage (full marks for 100\% statement coverage, nonlinearly scaled
down for less), 5 points for your tests having passing assertions that
verify the output. You will get 0 points if your code doesn't compile.

\section*{Questions 2--5: ``average''}
The next 4 questions are about the ``average'' Dart webapp, which I've
also included in the Vagrant image. I developed this webapp to help me
evaluate transfer requests. It is heavily based on the ``Tour of
Heroes'' Angular Dart demo, with additional help from the ``Nest o' Pirates'' 
Dart server example. The vagrant image includes
the webapp. The {\tt a1-technotes.pdf} document describes how to start
the webapp. Once started, you can navigate to your own copy of the
webapp at \url{http://localhost:8088/average} on your computer.

\section*{Question 2 (10 points)}
In this question, you will perform exploratory testing on ``average''.
The charter will be ``Explore the overall functionality of
average''. (1 point) Summarize in one or two sentences what you
perceive as the goal of ``average''. (5~points) Identify the tasks
that ``average'' should be able to do and classify them as primary or
contributing. (You probably want to do this in parallel with your
exploratory testing). (1 points) Identify areas of potential
instability. (3 points) Produce exploratory testing notes summarizing
your findings (one or two paragraphs); in Question 3 you will report a
bug, so no need to do that here.

\section*{Question 3 (10 points)}
(3 points) Identify a failure (bug) in the ``average'' webapp. (2 points) Write down a sequence of
steps to reproduce the bug. (5 points) Implement a fix for the bug and explain the
fix; show the incorrect and correct outputs. (Screenshots are probably
your best bet).

(Think about the program's input space and poke at corners of the
input space.)

\section*{Question 4 (10 points)}
(8 points) Draw a graph summarizing the navigation structure of ``average''.
Nodes are pages with distinct URLs. Edges are links between these
pages as realized by hyperlinks in the app. Submit your graph.
(2 points) Identify the dead end (node which has no successors) in the navigation
graph and submit a patch which corrects it.

\section*{Question 5 (10 points)}
The file {\tt shared/average/fake-data-small.sh}
contains some fake student and course
data. You can feed it to your instance of webapp by sourcing it: from a prompt 
(either inside or outside the VM),
type: ``\verb+. ./fake-data-small.sh+''

Your task is to evaluate the scalability of ``average''. (6 points) To do
so, programmatically generate test cases of various sizes. 

The scalability parameter to investigate is the number of students.
students.  
\begin{itemize}
\item Each fake student should give rise to one REST API call to
{\tt /student} and one to {\tt /student\_enrolment}, generating the
student and the enrolment in 1A. I recommend randomly drawing the
student's program from a small set that you manually create. 
You may also want to create fake names using a Faker library e.g.
\url{https://github.com/stympy/faker/blob/master/lib/locales/en.yml}.
\item Create a fixed set of courses. Mine are randomly generated but
yours don't have to be. 
\item For each student, generate fake course marks. 
You should generate the course marks randomly about a mean with a given
standard deviation.
\end{itemize}

Incidentally, SE 2021 courses had the following data: 

\begin{center}
\begin{tabular}{lrr}
& mean & $\sigma$\\
MATH 115 & 85.1 & 12.2 \\
MATH 117 & 83.0 & 12.6 \\
ECE 140 & 88.9 & 11.5 \\
ECE 105 & 82.2 & 13.0 \\
CS 137 & 82.5 & 10.6
\end{tabular}
\end{center}

(4 points) Empirically determine the scalability of the app; the easier way is to
manually time the JavaScript execution time (Chrome's developer tools
include a Timeline), and the harder (but better) way is to develop a
test harness.  Include the data points that you collect. Let's say
that a running time over 1 second is unacceptable.

\section*{Bonus (0 points)}
Figure out why the browser console displays a {\tt NullError} exception
when loading the ``Courses'' page and propose a fix.
I suspect this is also manifests as a user-visible failure which
you could use for Question 3.

\end{document}
