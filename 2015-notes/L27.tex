\documentclass[11pt]{article}
\usepackage{listings}
\usepackage{enumerate}
\usepackage{tikz}
\usepackage{alltt}
\usepackage{hyperref}
\usepackage{url}
%\usepackage{algorithm2e}
\usetikzlibrary{arrows,automata,shapes}
\tikzstyle{block} = [rectangle, draw, fill=blue!20, 
    text width=5em, text centered, rounded corners, minimum height=2em]
\tikzstyle{bt} = [rectangle, draw, fill=blue!20, 
    text width=1em, text centered, rounded corners, minimum height=2em]

\newtheorem{defn}{Definition}
\newtheorem{crit}{Criterion}
\newtheorem{claim}{Claim}
\newcommand{\strue}{\mbox{\scriptsize \sf true}}
\newcommand{\sfalse}{\mbox{\scriptsize \sf false}}
\newcommand{\true}{\mbox{\sf true}}
\newcommand{\false}{\mbox{\sf false}}

\newcommand{\handout}[5]{
  \noindent
  \begin{center}
  \framebox{
    \vbox{
      \hbox to 5.78in { {\bf Software Testing, Quality Assurance and Maintenance } \hfill #2 }
      \vspace{4mm}
      \hbox to 5.78in { {\Large \hfill #5  \hfill} }
      \vspace{2mm}
      \hbox to 5.78in { {\em #3 \hfill #4} }
    }
  }
  \end{center}
  \vspace*{4mm}
}

\newcommand{\lecture}[4]{\handout{#1}{#2}{#3}{#4}{Lecture #1}}
\topmargin 0pt
\advance \topmargin by -\headheight
\advance \topmargin by -\headsep
\textheight 8.9in
\oddsidemargin 0pt
\evensidemargin \oddsidemargin
\marginparwidth 0.5in
\textwidth 6.5in

\parindent 0in
\parskip 1.5ex
%\renewcommand{\baselinestretch}{1.25}

%\renewcommand{\baselinestretch}{1.25}
% http://gurmeet.net/2008/09/20/latex-tips-n-tricks-for-conference-papers/
\newcommand{\squishlist}{
 \begin{list}{$\bullet$}
  { \setlength{\itemsep}{0pt}
     \setlength{\parsep}{3pt}
     \setlength{\topsep}{3pt}
     \setlength{\partopsep}{0pt}
     \setlength{\leftmargin}{1.5em}
     \setlength{\labelwidth}{1em}
     \setlength{\labelsep}{0.5em} } }
\newcommand{\squishlisttwo}{
 \begin{list}{$\bullet$}
  { \setlength{\itemsep}{0pt}
     \setlength{\parsep}{0pt}
    \setlength{\topsep}{0pt}
    \setlength{\partopsep}{0pt}
    \setlength{\leftmargin}{2em}
    \setlength{\labelwidth}{1.5em}
    \setlength{\labelsep}{0.5em} } }
\newcommand{\squishend}{
  \end{list}  }
\begin{document}

\lecture{27 --- March 13, 2015}{Winter 2015}{Patrick Lam}{version 0}
Last time, we finished with an example showing that GACC does not
subsume PC.

\paragraph{Same example with CACC.} ~ \\[3em] % p = a \leftrightarrow b.
% a active: <a:T, b:T> -> p T, a active (since switching a to F => p = F)
%           <a:F, b:T> -> p F, a active (since switching a to T => p = T)
%
% b active: <a:T, b:T> -> p T, b active
%           <a:T, b:F> -> p F, b active
%
% test set { <a:T, b:T>, <a:F, b:T>, <a:T,b:F> } satisfies CACC.

So, how can we satisfy CACC on this example from last time?
\[ p = a \wedge (b \vee c) \]

% a active: b or c must be true; <a:T, b:T, c:F>, <a:F, b:F, c:T>
% b active: a T, c F: <a:T, b:F, c:F>, <a:T, b:T, c:F>
% c active: a T, b F: <a:T, b:F, c:F>, <a:T, b:F, c:T>
% does not satisfy RACC

~ \\[3em]

\paragraph{CACC versus RACC.} There's no clear reason to prefer
RACC over CACC. But we can construct situations where CACC is feasible
and RACC is infeasible. These situations typically involve
dependencies between clauses, making some combinations of clause
variables impossible.  See the book for an example.

RACC seems to come from the aviation community misinterpreting
a loosely-defined criterion known as ``MCDC''. RACC then corresponds to
``unique-cause MCDC'', but the FAA now allows CACC test suites, under the
name ``masking MCDC''.

\paragraph{Inactive Clause Coverage.} The book also defines the notion of
inactive clause coverage.

\paragraph{Exercise 1.} Consider predicate
\[ p = a \leftrightarrow b \wedge c.\]
\begin{enumerate}[(a)]
\item Identify all clauses in predicate $p$.
\item Compute and simplify conditions under which each of the clauses determines $p$.
  Conditions must use only $\wedge$, $\vee$ and $\neg$.
\item Write the complete truth table for all clauses. Label rows starting at 1.
  Row 1 is all-clauses-true. Include columns where each clause determines the
  predicate, and also a column for the predicate itself.
\item Identify all pairs of rows from your table that satisfy GACC with respect
  to each clause.
\item Identify all pairs of rows from your table that satisfy CACC with respect
  to each clause.
\item Identify all pairs of rows from your table that satisfy RACC with respect
  to each clause.
\end{enumerate}
\paragraph{Exercise 2.} Same exercise, but this time $p = a \wedge b$.

\section*{Infeasibility and Subsumption}
Of course, some test requirements are infeasible. It's easy to make RACC infeasible.

\paragraph{Workaround I.} Satisfy feasible TRs, drop infeasible TRs.

\paragraph{Workaround II.} If you can't satisfy a particular
coverage criterion, use a looser criterion. For instance, if you can't satisfy
RACC, settle for CACC.

\paragraph{Causes of infeasibility.} From the
definition, a logic coverage test requirement $t$ is infeasible if
no test case can satisfy test requirement $t$. Here's an example of
a case where predicate coverage is infeasible:

\begin{lstlisting}
  void m(char[] c) {
    if (c.length < 0) {
      print ("infeasible");
   }
  }
\end{lstlisting}

Clearly, no array $c$ can have negative length, so the predicate can't
be false.

I can think of the following ways for a coverage test requirement to 
be infeasible; these ways reflect the steps you need to follow to satisfy
test requirements.
\begin{itemize}
\item The predicate is unreachable in its context (as in graph reachability); or
\item The method containing the predicate never assigns appropriate values
to its variables to cause the desired clause values (as in the example above); or
\item A clause never determines a predicate and you are trying to achieve an active clause coverage criterion.
\end{itemize}

\clearpage

\section*{Subsumption Chart}
Without inactive clause criteria, we can simplify the
subsumption chart in the book to:

\tikzstyle{bw} = [rectangle split, rectangle split parts=2, draw, fill=blue!20, 
    text width=8em, text centered, rounded corners, minimum height=2em]

\begin{center}
\scalebox{0.7}{
\begin{minipage}{.4\textwidth}
  \label{SC}
\begin{tikzpicture}[->,>=stealth',shorten >=1pt,auto,node distance=20mm,
                    semithick,initial text=]

  \node[bw]   (CoC)               {Combinatorial Clause Coverage \nodepart{second} CoC};
  \node[bw]   (RACC) [below of=CoC] {Restricted Active Clause Coverage \nodepart{second} RACC};
  \node[bw]   (CACC) [below of=RACC] {Correlated Active Clause Coverage \nodepart{second} CACC};
  \node[bw]   (GACC) [below of=CACC] {General Active Clause Coverage \nodepart{second} GACC};
  \node[bw]   (CC) [below of=GACC] {Clause Coverage \nodepart{second} CC};
  \node[bw]   (PC) [right of=CC,xshift=5em] {Predicate Coverage \nodepart{second} PC};

  \path (CoC) edge node {} (RACC)
        (RACC) edge node {} (CACC)
        (CACC) edge node {} (GACC)
               edge[bend left] node {} (PC)
        (GACC) edge node {} (CC);

\end{tikzpicture}
\end{minipage}
}
\end{center}

\section*{How to get Logic Coverage}
To achieve logic coverage:
\squishlist
\item identify the predicates $p \in P$ in the program fragment under test;
\item figure out how to reach each of the predicates;
\item make $c$ determine $p$ (for the active clause criteria); and
\item find values for (program) variables to meet various criteria.
\squishend

We've seen how to make $c$ determine $p$; let's look at an example where
we find values for variables to meet the criteria.

\paragraph{Predicates.} We'll use the following predicate, modified from the textbook example
{\tt TestPat}:
\begin{verbatim}
    isPat == false && iSub + pL - 1 < sL || subject[iSub] == pattern[0]
\end{verbatim}
We can assign names to these clauses, giving the symbolic predicate:\\[1em]
%\[ a \wedge b \vee c. \]

\paragraph{Determination.} We analyze the predicate and determine the conditions under which each clause
determines the predicate: \\[3em]
% a determines when b or c is true; b determines when c is false, a true;
% c determines when b is false, a true.

\paragraph{Mapping Values to Predicates.} Now we need to find program values that correspond to different
predicate values. {\tt isPat} is a boolean variable, so it's easy. The
code says that {\tt pL} and {\tt sL} are lengths of argument arrays,
so we need to assign arrays such that {\tt iSub + pL - 1 < sL}.
Finally, we can assign values to {\tt iSub} and {\tt
  subject} such that {\tt subject[iSub]} is equal to or different from
{\tt pattern}, as needed. Examples:\\[3em]

%{\tt iSub = 0, pattern = new char[1], subject = new char[5]} ($b$ true)
%{\tt iSub = 0, pattern[0] = subject[0] = 'a'}.

\paragraph{Coverage Criteria.} We now list the coverage criteria. For PC,
we have: \\[1.5em]

CC:\\[1.5em]

CoC:\\[2em]

GACC:\\[5em]

(What are all of the ways to satisfy GACC?)

CACC: additional constraint that each pair must cause $p$ to be
both true and false.\\[3em]

RACC:

\paragraph{Concluding Comments on Logic Criteria.} GACC doesn't imply
predicate coverage, but often does when 3 or more terms are involved.
RACC can be unsatisfiable. CACC seems to often be ``just right''.

\end{document}
