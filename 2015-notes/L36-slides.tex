\documentclass{beamer}

\usetheme{Boadilla}

%\includeonlyframes{current}

\usepackage{times}
\usefonttheme{structurebold}
\usepackage{listings}
\usepackage{ragged2e}

\usepackage{pgf}
\usepackage{tikz}
\usepackage{alltt}
\usepackage[normalem]{ulem}
\usetikzlibrary{arrows}
\usetikzlibrary{automata}
\usetikzlibrary{shapes}
\usepackage{amsmath,amssymb}
\usepackage{rotating}
\usepackage{ulem}
\usepackage{listings}
\usepackage{enumerate}
\usepackage{tikz}
\tikzset{
  every overlay node/.style={
    draw=black,fill=white,rounded corners,anchor=north west,
  },
}
\def\tikzoverlay{%
   \tikz[baseline,overlay]\node[every overlay node]
}%

%\setbeamercovered{dynamic}
\setbeamertemplate{footline}[page number]{}
\setbeamertemplate{navigation symbols}{}
\usefonttheme{structurebold}

\title{Software Testing, Quality Assurance \& Maintenance---Lecture 36}
\author{Patrick Lam}
\date{April 6, 2015}

\colorlet{redshaded}{red!25!bg}
\colorlet{shaded}{black!25!bg}
\colorlet{shadedshaded}{black!10!bg}
\colorlet{blackshaded}{black!40!bg}

\colorlet{darkred}{red!80!black}
\colorlet{darkblue}{blue!80!black}
\colorlet{darkgreen}{green!80!black}

\newcommand{\rot}[1]{\rotatebox{90}{\mbox{#1}}}
\newcommand{\gray}[1]{\mbox{#1}}

\newenvironment{changemargin}[1]{% 
  \begin{list}{}{% 
    \setlength{\topsep}{0pt}% 
    \setlength{\leftmargin}{#1}% 
    \setlength{\rightmargin}{1em}
    \setlength{\listparindent}{\parindent}% 
    \setlength{\itemindent}{\parindent}% 
    \setlength{\parsep}{\parskip}% 
  }% 
  \item[]}{\end{list}}


\lstset{ %
language=C++,
basicstyle=\ttfamily,commentstyle=\scriptsize\itshape,showstringspaces=false,breaklines=true}

\begin{document}

\begin{frame}
  \titlepage
\end{frame}

\begin{frame}
  \frametitle{Summary: Useful Terms}
  \Large
  \begin{changemargin}{1cm}
\begin{itemize}
\item software faults, errors and failures; 
\item tests and test requirements; and
\item coverage criteria and subsumption.
\end{itemize}
  \end{changemargin}
\end{frame}

\begin{frame}
  \frametitle{Summary: Theory}
  \Large
  \begin{changemargin}{1.5cm}
Key Coverage Criteria:
\begin{itemize}
\item graph coverage;
\item logic coverage;
\item syntax-based coverage \\ \hspace*{1cm} (and mutation and fuzzing); and 
\item input-space coverage.
\end{itemize}
Evaluate test suites against these criteria.\\
\hspace*{1cm}(guides test suite construction.)
  \end{changemargin}
\end{frame}

\begin{frame}
  \frametitle{Summary: Practice}
  \begin{changemargin}{1.8cm}
We gained experience with:
\begin{itemize}
\item using tools for unit testing (JUnit) \& \\ \hspace*{1cm} bug detection (Valgrind) 
\item  test design: \\ ~~testability, badly designed tests, self-checking tests;
\item writing an invariant detection and bug detection tool (using
LLVM and the idea behind Coverity);\\
\item using the state-of-the-art tool Coverity for bug finding;
\item writing good bug reports.
\end{itemize}
and learned about automated testing and bug detection tools,
regression testing, testing for concurrency (Helgrind), and
state-of-the-art techniques (Daikon, Coverity, iComment).
  \end{changemargin}
\end{frame}

\begin{frame}
  \frametitle{About the Final}
  \Large
  \begin{changemargin}{1.5cm}
\begin{itemize}
\item Tuesday, April 21, 2015, 12:30-3:00, PAC.
\item Open-book, open note exam.
\item You may consult any printed material
(books, slides, notes, etc).
\item No electronic devices.
\end{itemize}
  \end{changemargin}
\end{frame}

\begin{frame}
  \frametitle{Review: Input Space Partitioning}
  \large
  \begin{changemargin}{2cm}
\begin{itemize}
\item Can't feed all inputs to the program---\\ test a representative set.
\item Input space partitioning makes this idea more
formal: test one input from each partition.
\item Two properties for partitions:\\

\hspace*{1cm} Completeness\\
\hspace*{1cm} Disjointness
\end{itemize}
  \end{changemargin}
\end{frame}

\begin{frame}
  \frametitle{Review: Input Domain Models}
  \Large
  \begin{changemargin}{1.7cm}
Require creativity and
analysis to formulate.

Two general approaches:
\begin{itemize}
\item Interface-based, using the input space directly, or
\item Functionality-based, using a functional or behavioural
view of the program.
\end{itemize}
  \end{changemargin}
\end{frame}

\begin{frame}[fragile]
  \frametitle{Review: Cross-checking program beliefs (MUST)}
  \begin{changemargin}{1cm}
\begin{itemize}
\item MUST beliefs:
inferred from acts that imply beliefs code {\bf must} have.
\begin{lstlisting}
x = *p / z; 
// MUST belief: p not null
// MUST: z != 0
unlock(l);
// MUST: l acquired
x++;
// MUST: x not protected by l
\end{lstlisting}

 Check using internal consistency: \\
infer beliefs at different
locations, then cross-check for contradiction.
\end{itemize}
  \end{changemargin}
\end{frame}

\begin{frame}[fragile]
  \frametitle{Review: Cross-checking program beliefs (MAY)}
  \begin{changemargin}{1cm}
\begin{itemize}
\item MAY beliefs: 
Could be coincidental.

Inferred from acts that imply beliefs code {\bf may} have.
\begin{lstlisting}
A(); A(); A(); A();
...  ...  ...  ... 
B(); B(); B(); B();
// MAY: A() and B()
// must be paired
\end{lstlisting}

Check as MUST beliefs; rank errors by belief confidence.
\end{itemize}
  \end{changemargin}
\end{frame}

\begin{frame}
  \frametitle{Review: Static vs Dynamic Analysis}
  \begin{changemargin}{2cm}

Static analysis:
\begin{itemize}
\item Conceptually, can find everything.
\item Problems: pointer aliasing, false positives.
\end{itemize}

Dynamic analysis:
\begin{itemize}
\item Imposes run-time overhead.
\item Depends on the quality of inputs/environments.
\item May report false negatives.
\end{itemize}
  \end{changemargin}
\end{frame}

\begin{frame}
  \frametitle{Review: Race conditions and Deadlocks}
  \begin{changemargin}{2cm}
Race condition: \\
two concurrent
accesses to the same state, \\
at least one of
which is a write.\\[1em] 

Deadlock: \\ 
cyclic structure between lock acquisitions\\
 that 
will never succeed.
  \end{changemargin}
\end{frame}

\begin{frame}
  \frametitle{Review: Testing Concurrent Programs}
  \begin{changemargin}{2cm}
Some options:
\begin{itemize}
\item Run them multiple times.
\item Add noise: sleep, background load, \ldots.
\item Use tools: Helgrind, etc.
\item Force different scheduling.
\end{itemize}

Some of these options use the following technologies:
\begin{itemize}
\item lock-set;
\item happens-before;
\item other state-of-art techniques.
\end{itemize}
  \end{changemargin}
\end{frame}

\begin{frame}
  \frametitle{Review: Regression Tests}
  \begin{changemargin}{2cm}
\large
\begin{itemize}
\item Gotta automate them.
\item Have enough regression tests:
\begin{itemize}
\item Too few: you'll miss bugs.
\item Too many: maintenance overhead, and if badly designed, too slow to run.
\end{itemize}
\item Keep them up-to-date.
\end{itemize}
  \end{changemargin}
\end{frame}

\begin{frame}
  \frametitle{Review: Output validation}
  \begin{changemargin}{2cm}
\Large
How do you know your test outputs are correct?
\begin{itemize}
\item Hope for the best.
\item Manual effort.
\item Compute multiple ways.
\item Checksums, redundant data, etc.
\end{itemize}
  \end{changemargin}
\end{frame}

\begin{frame}
  \frametitle{Review: Syntax-based Testing}

{\scriptsize
\begin{tabular}{l|ll}
& Program-based & Input Space \\ \hline
Grammar & Programming language & Input languages / XML \\
Summary & Mutates programs / tests integration & Input space testing \\
Use Ground String? & Yes (compare outputs) & No \\
Use Valid Strings Only? & Yes (mutants must compile) & Invalid only \\
Tests & Mutants are not tests & Mutants are tests \\
Killing & Generate tests by killing & Not applicable \\
\end{tabular}
}

  \begin{changemargin}{2cm}
Strong and weak mutants.
  \end{changemargin}
\end{frame}

\begin{frame}
  \frametitle{Sample Questions}

\begin{changemargin}{1cm}
\begin{itemize}
\item Give an example of a prime path.
\item Exhibit a case where a clause always determines
a predicate.
\item Exhibit a case where a clause never determines
a predicate.
\item Why might GACC be infeasible?
\item What causes node coverage to be infeasible?
\item Give an example of a $du$-path.
\item Given a function, draw the CFG. Identify the def-set and use-set of each node, 
and list the TR for ADC and AUC.
\item The exercise questions seen in class.
\end{itemize}
\end{changemargin}
\end{frame}

\begin{frame}
  \frametitle{More Sample Questions}
\begin{changemargin}{1cm}
\begin{itemize}
\item Read given program, give an input that
strongly kills this mutant.
\item Enumerate requirements for Prime Path
Coverage on given example. If PPC is feasible,
provide a test set which achieves it. Otherwise,
provide the test set that you feel comes closest
to achieving PPC.
\end{itemize}
\end{changemargin}
\end{frame}


\begin{frame}[fragile]
  \frametitle{Input Space Partitioning: Sample Questions}
\begin{changemargin}{1cm}
\begin{itemize}
\item Propose an input space partition for this method
(use a functionality-based input domain model):
\begin{lstlisting}
static boolean isPrime (int n) {
  if (n<=1) return false;
  for (int i = 2; i <= (int)(Math.sqrt(n)); i++)
    if (n % i == 0)
      return false;
  return true;
}
\end{lstlisting}
\end{itemize}
\end{changemargin}
\end{frame}

\begin{frame}
\begin{center}\Huge
Best of luck\\
 \alert{\sout{after Convocation}} \\ 
on your work term.
\end{center}
\end{frame}

\end{document}
