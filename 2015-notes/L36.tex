\documentclass[11pt]{article}
\usepackage{listings}
\usepackage{tikz}
\usepackage{url}

%\usepackage{algorithm2e}
\usetikzlibrary{arrows,automata,shapes}
\tikzstyle{block} = [rectangle, draw, fill=blue!20, 
    text width=2.5em, text centered, rounded corners, minimum height=2em]
\tikzstyle{bw} = [rectangle, draw, fill=blue!20, 
    text width=3.5em, text centered, rounded corners, minimum height=2em]

\newcommand{\handout}[5]{
  \noindent
  \begin{center}
  \framebox{
    \vbox{
      \hbox to 5.78in { {\bf ECE459: Programming for Performance } \hfill #2 }
      \vspace{4mm}
      \hbox to 5.78in { {\Large \hfill #5  \hfill} }
      \vspace{2mm}
      \hbox to 5.78in { {\em #3 \hfill #4} }
    }
  }
  \end{center}
  \vspace*{4mm}
}

\newcommand{\lecture}[4]{\handout{#1}{#2}{#3}{#4}{Lecture #1}}
\topmargin 0pt
\advance \topmargin by -\headheight
\advance \topmargin by -\headsep
\textheight 8.9in
\oddsidemargin 0pt
\evensidemargin \oddsidemargin
\marginparwidth 0.5in
\textwidth 6.5in

\parindent 0in
\parskip 1.5ex
%\renewcommand{\baselinestretch}{1.25}

\lstset{basicstyle=\ttfamily \scriptsize,
  basicstyle=\ttfamily,
   columns=fullflexible,
   breaklines=true,
   numbers=left,
   numberstyle=\scriptsize,
   stepnumber=1,
   mathescape=false,
   tabsize=2,
   showstringspaces=false
}

\newenvironment{itemizep}{
 \begin{itemize}
  \setlength{\itemsep}{0pt}
  \setlength{\parsep}{3pt}
  \setlength{\topsep}{3pt}
  \setlength{\partopsep}{0pt}
  \setlength{\leftmargin}{1.5em}
  \setlength{\labelwidth}{1em}
  \setlength{\labelsep}{0.5em} }
 {\end{itemize}}

\begin{document}

\lecture{36 --- April 6, 2015}{Winter 2015}{Patrick Lam}{version 0}

\section*{Ensuring Testability}
This is from Chapter 6 of the Meszaros book.

Retrofitting testability is harder than it needs to be. So is
designing testability into a system that doesn't yet exist. The happy
medium is in between: incorporate testability as you go along.  Build
your system for testability.  Plus, tests help during development.

\paragraph{Control Points and Observation Points.}
We previously saw these concepts in Lecture 4. \emph{Control points} are ways to get the 
system under test (SUT) to do something. It is OK to have test-only control points, although
you should probably label them as such. \emph{Observation points} are how you figure out
if the system under test is doing the right thing. Direct observation points are clear 
and mainly consist of return values. But there are also indirect observation points,
where you add test doubles and have them check that the SUT is doing the right thing,
for instance by observing outbound calls.

\paragraph{Interaction Styles and Testability Patterns.}
The easiest pattern is the \emph{round-trip test}: your test uses the public interface only.
This works great when your system is sufficiently controllable or observable.

An alternative is the \emph{layer-crossing test}: you use the SUT's
API, but then you watch what it does using test spies or mock
objects. The danger is that you might get overspecified
software. Sometimes it doesn't matter that the SUT calls {\tt x()} and
then {\tt y()}; it's fine to change the order. Your test should not
tie the hands of the implementer.

\emph{Dependency Injection} is a good way to make mock objects be used
by the SUT. More about that shortly. Or, you can use test-specific subclasses,
or, as a very last resort, test hooks. They are perilously close to having
test code in production, though.

Finally, you might have a \emph{asynchronous test}. You send messages
to the system under test and wait for responses. Waiting can be slow
and sometimes unreliable, though. This also includes using the UI to
test, which is painful and fragile. It's better to interact with the
underlying system directly, when possible.

\paragraph{More on Dependency Injection.} This has always been confusing to
me. So, here's an example. Consider:

\begin{lstlisting}
public void testDisplayCurrentTime_AtMidnight() {
  TimeDisplay sut = new TimeDisplay();
  String result = sut.getCurrentTimeAsHtmFragment();
  String expectedTimeString = "<span class=\"...\">Midnight</span>";
  assertEquals(expectedTimeString, result);
}

public String getCurrentTimeAsHtmlFragment() {
  Calendar currentTime;
  try {
    currentTime = new DefaultTimeProvider.getTime();
  } catch (Exception e) {
    return e.getMessage();
  } // etc.
}
\end{lstlisting}

This is not going to succeed very often! The dependency on
{\tt DefaultTimeProvider} is hard-coded. You can instead 
introduce a {\tt TimeProvider} argument. There are lots of choices:
you can introduce it as a function parameter, or in the constructor,
or using a setter method. Or, you can have dependency lookup.

\end{document}
