\documentclass[11pt]{article}
\usepackage{listings}
\usepackage{tikz}
\usepackage{enumerate}
\usepackage{url}
%\usepackage{algorithm2e}
\usetikzlibrary{arrows,automata,shapes}
\tikzstyle{block} = [rectangle, draw, fill=blue!20, 
    text width=5em, text centered, rounded corners, minimum height=2em]
\tikzstyle{bt} = [rectangle, draw, fill=blue!20, 
    text width=1em, text centered, rounded corners, minimum height=2em]

\lstset{ %
language=Java,
basicstyle=\ttfamily,commentstyle=\scriptsize\itshape,showstringspaces=false,breaklines=true,numbers=left}

\newtheorem{defn}{Definition}
\newtheorem{crit}{Criterion}

\newcommand{\handout}[5]{
  \noindent
  \begin{center}
  \framebox{
    \vbox{
      \hbox to 5.78in { {\bf Software Testing, Quality Assurance and Maintenance } \hfill #2 }
      \vspace{4mm}
      \hbox to 5.78in { {\Large \hfill #5  \hfill} }
      \vspace{2mm}
      \hbox to 5.78in { {\em #3 \hfill #4} }
    }
  }
  \end{center}
  \vspace*{4mm}
}

\newcommand{\lecture}[4]{\handout{#1}{#2}{#3}{#4}{Lecture #1}}
\topmargin 0pt
\advance \topmargin by -\headheight
\advance \topmargin by -\headsep
\textheight 8.9in
\oddsidemargin 0pt
\evensidemargin \oddsidemargin
\marginparwidth 0.5in
\textwidth 6.5in

\parindent 0in
\parskip 1.5ex
%\renewcommand{\baselinestretch}{1.25}

\begin{document}

\lecture{3 --- January 9, 2017}{Winter 2017}{Patrick Lam}{version 2}

\paragraph{Exercise: findLast.} Here's a faulty program.
\begin{lstlisting}
    static public int findLast(int[] x, int y) {
        /* bug: loop test should be i >= 0 */
        for (int i = x.length - 1; i > 0; i--) {
            if (x[i] == y) {
                return i;
            }
        }
        return -1;
    }
\end{lstlisting}
You might expect to be asked a question like this:
\begin{enumerate}[(a)]
    \item Identify the fault, and fix it.
    \item If possible, identify a test case that does not execute the fault.
    \item If possible, identify a test case that executes the fault, but does not result in an error state.
    \item If possible, identify a test case that results in an error, but not a failure. (Hint: program counter)
    \item For the given test case, identify the first error state. Be sure to describe the complete state.
\end{enumerate}

I asked you to work on that question in class. Here are some answers:

\begin{enumerate}[(a)]
\item The loop condition must include index 0: {\tt i >= 0}.
\item This is a bit of a trick question. To avoid the loop condition, you must not enter the loop body. The only way to do that is to pass in {\tt x = null}. You should also state, for instance, {\tt y = 3}. The expected output is a {\tt NullPointerException}, which is also the actual output.
\item Inputs where {\tt y} appears in the second or later position execute the fault but do not result in the error state; nor do inputs where {\tt x} is an empty array. (There may be other inputs as well.) So, a concrete input: {\tt x = \{2, 3, 5\}; y = 3}. Expected output = actual output = 1.
\item One error input that does not lead to a failure is where {\tt y}
  does not occur in {\tt x}. That results in an incorrect PC after the final executed iteration of the loop.
 \item After running line 6 with {\tt i = 1}, the decrement occurs, followed by the evaluation of {\tt i > 0}, causing the PC to exit the loop (statement 8) instead of returning to statement 4. The faulty state is {\tt x = \{2, 3, 5\}; y = 3; i = 0; PC = 8}, while correct state would be {\tt PC = 4}.
\end{enumerate}

Someone asked about distinguishing errors from failures. These
questions were about failures at the method level, and so a wrong
return value would be a failure when we're asking about methods. In
the context of a bigger program, that return value might not be
visible to the user, and so it might not constitute a failure, just an
error.

\section*{Line Intersections}
We then talked about different ways of validating a test suite.
Consider the following Python code, found by Michael Thiessen
on stackoverflow (\url{http://stackoverflow.com/questions/306316/determine-if-two-rectangles-overlap-each-other}).

\begin{lstlisting}[language=python]
class LineSegment:
    def __init__(self, x1, x2):
        self.x1 = x1; self.x2 = x2;

def intersect(a, b):
    return (a.x1 < b.x2) & (a.x2 > b.x1);
\end{lstlisting}

We could construct test suites that:
\begin{itemize}
\item execute every statement in {\tt intersect} (statement coverage). Well, that's not very useful; any old test case
  will do that. There are no branches, so what we'll call edge coverage
  doesn't help either.
\item feed random inputs to {\tt intersect}; unfortunately,
  interesting behaviours are not random, so it won't help much
  in general.
\item check all outputs of {\tt intersect} (i.e. a test case with
  lines that intersect and one with lines that don't intersect): we're
  getting somewhere---that will certify that the method works in some
  cases, but it's easy to think of situations that we missed.
\item check different values of clauses {\tt a.x1 < b.x2} and {\tt a.x2 > b.x1} (logic coverage)---better than the above coverage criteria, but still misses interesting behaviours;
  \item analyze possible inputs and cover all interesting combinations of inputs (input space coverage)---can create an exhaustive test suite, if your analysis is sound.
\end{itemize}

Let's try to prove correctness of {\tt intersect}. There's an old
saying about testing---supposedly, it can only find the presence of
bugs, not their absence. This is not completely true, especially if
you have reliable software that automatically constructs an exhaustive
test suite.  But that is beyond the state of the practice in 2015, for
the most part.

\paragraph{Inputs to {\tt intersect}.} There are essentially
four inputs to this function. Rename them $aAbB$, for {\tt a.x1},
{\tt a.x2}, etc.
\begin{itemize}
\item Let's first assume that all points are distinct. We should
  make a note to ourselves to check violations of this, as well:
  we may have $a = A, a = b, a = B$, and symmetrically for $B$:
  $b = a, b = A, b = B$.
  \item For the purpose of the analysis, let's assume that $a < b$;
    when constructing test cases, we can swap $a$ and $b$ around.
    That's why there are duplicate assert statements below.
  \item Without loss of generality, we can assume that
$a < A$ and $b < B$. (We ought to update the constructor if we
    want to make that assumption.) 
\end{itemize}
With these assumptions, we have to test the three possible permutations
$aAbB$, $abAB$, and $abBa$. It is simple to construct test cases
for these permutations, using Python's unittest framework:
\begin{lstlisting}
      def test_aAbB(self):
        a = LineSegment(0,2)
        b = LineSegment(3,7)
        self.assertFalse(intersect(a,b))
        self.assertFalse(intersect(b,a))

    def test_abAB(self):
        a = LineSegment(0,4)
        b = LineSegment(3,7)
        self.assertTrue(intersect(a,b))
        self.assertTrue(intersect(b,a))

    def test_abBA(self):
        a = LineSegment(0,4)
        b = LineSegment(1,2)
        self.assertTrue(intersect(a,b))
        self.assertTrue(intersect(b,a))
\end{lstlisting}
Those test cases pass. However, if you construct test cases for
equality (as I've committed to the repository), you see that the given {\tt
  intersect} function fails on line segments that intersect only at a
point. Replacing \verb+<+ with \verb+<=+ and \verb+>+ with \verb+>=+
fixes the code.

\end{document}
