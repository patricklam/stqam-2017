\documentclass[11pt]{article}
\usepackage{listings}
\usepackage{tikz}
%\usepackage{algorithm2e}
\usetikzlibrary{arrows,automata,shapes}
\tikzstyle{block} = [rectangle, draw, fill=blue!20, 
    text width=5em, text centered, rounded corners, minimum height=2em]
\tikzstyle{bt} = [rectangle, draw, fill=blue!20, 
    text width=1em, text centered, rounded corners, minimum height=2em]

\newtheorem{defn}{Definition}
\newtheorem{crit}{Criterion}
\newcommand{\true}{\mbox{\sf true}}
\newcommand{\false}{\mbox{\sf false}}

\newcommand{\handout}[5]{
  \noindent
  \begin{center}
  \framebox{
    \vbox{
      \hbox to 5.78in { {\bf Software Testing, Quality Assurance and Maintenance } \hfill #2 }
      \vspace{4mm}
      \hbox to 5.78in { {\Large \hfill #5  \hfill} }
      \vspace{2mm}
      \hbox to 5.78in { {\em #3 \hfill #4} }
    }
  }
  \end{center}
  \vspace*{4mm}
}

\newcommand{\lecture}[4]{\handout{#1}{#2}{#3}{#4}{Lecture #1}}
\topmargin 0pt
\advance \topmargin by -\headheight
\advance \topmargin by -\headsep
\textheight 8.9in
\oddsidemargin 0pt
\evensidemargin \oddsidemargin
\marginparwidth 0.5in
\textwidth 6.5in

\parindent 0in
\parskip 1.5ex
%\renewcommand{\baselinestretch}{1.25}

\begin{document}

\lecture{32 --- March 25, 2015}{Winter 2015}{Patrick Lam}{version 0}

\section*{Combining Characteristics}
We next investigate how to test multiple inputs/characteristics, each with 
their own partitions.

Consider three inputs, with the following representative values from 
each block of a partition:

\[ [A, B] \qquad [1, 2, 3] \qquad [x, y] \]

Then the obvious exhaustive coverage criterion would be:

\begin{crit}
{\bf All Combinations Coverage} (ACoC). TR contains all combinations of
blocks from all characteristics.
\end{crit}
and we could start enumerating test cases: $(A, 1, x), (A, 1, y), (A, 2, x),
\ldots$; there would be 12 tests in general, because the number of
test requirements is the product
\[ \prod^Q B_i \]
This can get too big if there are too many characteristics, so we'll
propose heuristics. For instance, we might try each
block at least once.

\begin{crit}
{\bf Each Clause Coverage} (ECC). TR contains the requirement to include
one value from each block for each characteristic in some test case.
\end{crit}
The following test suite would work: $(A, 1, x), (B, 2, y), (A, 3, x)$.
This criterion is analogous to clause coverage, and requires
at least as many tests as the largest number of blocks in a partition.

On the TriTyp interface-based IDM, we can satisfy ECC with four tests:
$(2, 2, 2), (1, 1, 1), (0, 0, 0)$ and $(-1, -1, -1)$; note that these 
tests don't tell you much.

\paragraph{Combining Values.} Imposing more test requirements
might require more interesting test cases. 

\begin{crit}
{\bf Pair-Wise Coverage} (PWC). TR contains the requirement to combine
a value for each block for each characteristic with some value from 
every other block for other characteristics.
\end{crit}

PWC imposes 16 test requirements for our running example:
\[ (A, 1), (A, 2), (A, 3), (A, x), (A, y), (B, 1), (B, 2), \ldots, (3, y), \]
and we can satisfy these requirements with 8 test cases:
\[ (A, 1, x), (A, 2, x), (A, 3, x), (A, *, y), (B, 1, y), (B, 2, y),
(B, 3, y), (B, *, x). \]
(The book contains a bogus analysis of the test set size; it actually
analyzes \# of test requirements.)

\begin{crit}
{\bf T-Wise Coverage} (TWC). TR contains the requirement to combine
a value for each block for each characteristic with $T$ values from 
 other blocks for other characteristics.
\end{crit}
When $T$ is the number of partitions, TWC is the same as ACoC.
$T>2$ doesn't seem to help much.

\paragraph{Base Choice Coverage.} We've treated all blocks as 
equivalent so far. We'll now pick one block as the most important
block and call it the base choice. Examples: simplest, smallest,
first (with respect to some ordering), most likely.

\begin{crit}
{\bf Base Choice Coverage} (BCC). Choose a base choice block for each
characteristic, and a base test by combining the base choices for all
characteristics. For each characteristic $c$, TR contains the
requirement for a test which varies the base test by using all other
blocks for characteristic $c$.
\end{crit}

In our example, suppose the base choice blocks are $A$, $1$ and $x$. Then 
the base choice test is $(A, 1, x)$. Other tests would be $({\bf B}, 1, x), 
(A, {\bf 2}, x), (A, {\bf 3}, x), (A, 1, {\bf y})$.

Number of tests:
\[ 1 + \sum^Q (B_i - 1) \]
where $B_i$ is the number of blocks for characteristic $i$. 

For TriTyp,
we have $1 + 3 + 3 + 3 = 10$ tests. Say that the block ``$> 1$'' is the base
choice. A base choice could be $(2, 2, 2)$ and the non-base choices might be:
\[ (2, 2, {\bf 1}), (2, 2, {\bf 0}), (2, 2, {\bf -1}), (2, {\bf 1}, 2), 
(2, {\bf 0}, 2), (2, {\bf -1}, 2), ({\bf 1}, 2, 2), ({\bf 0}, 2, 2),
({\bf -1}, 2, 2).\]

BCC is like ECC except that you're only allowed to change 1 thing
at a time from the base. We can also allow more than 1 base choice:

\begin{crit}
{\bf Multiple Base Choice Coverage} (MBCC). Choose one or more base
choice blocks for each characteristic, and form base tests by using
each base choices for each characteristic at least once. For each
characteristic $c$, TR contains the requirement for a test which
varies each base test by using all other blocks for characteristic $c$.
\end{crit}

For example, we could have two base choices for side 1 in TriTyp, 
$> 1$ and $= 1$. This gives the base tests $(2, 2, 2)$ and $(1, 2, 2)$.
Then we proceed as in plain BCC, but using each of the two base tests.

Upper bound on number of tests:
\[ M + \sum^Q M (B_i - m_i) \]
where $M$ is the number of base tests and $M_i$ is the number of base
choices for characteristic $i$.

\section*{Subsumption Chart}

\tikzstyle{bw} = [rectangle split, rectangle split parts=2, draw, fill=blue!20, 
    text width=8em, text centered, rounded corners, minimum height=2em]

\begin{center}
\label{SC}
\begin{tikzpicture}[->,>=stealth',shorten >=1pt,auto,node distance=30mm,
                    semithick,initial text=]

  \node[bw]   (ACoC)               {All Combinations \nodepart{second} ACoC};
  \node[bw]   (TWC) [below left of=ACoC,yshift=5mm] {$T$-Wise \nodepart{second} TWC};
  \node[bw]   (PWC) [below of=TWC,yshift=10mm] {Pair-Wise \nodepart{second} PWC};
  \node[bw]   (ECC) [below right of=PWC,yshift=5mm] {Each Choice \nodepart{second} ECC};
  \node[bw]   (MBCC) [below right of=ACoC,yshift=5mm] {Multiple Base Choice \nodepart{second} MBCC};
  \node[bw]   (BCC) [below of=MBCC,yshift=10mm] {Base Choice \nodepart{second} BCC};

  \path (ACoC) edge node {} (TWC) edge node {} (MBCC)
        (TWC) edge node {} (PWC)
        (PWC) edge node {} (ECC)
        (MBCC) edge node {} (BCC)
        (BCC) edge node {} (ECC);

\end{tikzpicture}
\end{center}

\section*{Constraints Among Partitions}
Picking choices can lead to infeasible test requirements.
For example, it's impossible to find test cases for scalene triangles
with negative side lengths, because the negative side length forces the
classification ``invalid''. Similarly, ``{\tt containsElement} returns true''
requires the input to satisfy ``element non-null''.

\paragraph{Handling infeasible TRs.} We can simply drop infeasible TRs,
for ACoC, PWC, and TWC. For BCC, we can change the base case, e.g. if
we choose base case ``scalene'', then we can't test $S1 < 0$. Instead,
we can add an additional base case or modify the base case to
``invalid''.

\paragraph{Sort example.} A sort routine might take an array
and could return a sorted array, the largest element and the smallest
element.

Possible characteristics:\\[5em]
% length of array, type of elements, position of maximum value,
% position of minimum value, max val, min val

Potential partitioning:\\[5.5em]
% length: {0, 1, 2..100, 101..maxint}
% type: {int, char, string, Comparable, other}
% max: {<= 0, 1, > 1, 'a', 'Z', 'b', ..., 'Y', blank, nonblank}
% min: same
% max position: {1, 2..length-1, length}
% min position: same

Not all combinations are possible.

\end{document}
